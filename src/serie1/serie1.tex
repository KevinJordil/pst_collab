\documentclass[11pt,a4paper]{report}

%
% Packages
%
\usepackage[T1]{fontenc}
\usepackage[utf8]{inputenc}
\usepackage{amsmath,amsthm,amssymb}
\usepackage[svgnames]{xcolor}
\usepackage[english,french]{babel}
\usepackage{multicol}
\usepackage{pstricks,pst-node,pst-text,pst-poly,pst-3d}
\usepackage{enumitem}
\usepackage{sidenotes}
\usepackage{graphicx} % Required to insert images
\usepackage{array}
\usepackage{booktabs} % Required for better horizontal rules in tables


%
% Constant and Variables
%
\setlength{\textwidth}{175mm}
\setlength{\textheight}{255mm}
\setlength{\oddsidemargin}{-10mm}
\setlength{\topmargin}{-15mm}
\setlength{\parskip}{0.2cm}

\definecolor{vertfonce}{rgb}{0,0.5,0}

%
% Commands
%
\newcommand{\ds}{\displaystyle}
\newcommand{\scr}{\scriptscriptstyle}
\newcommand{\bs}[1]{\ensuremath{\boldsymbol{#1}}}
\renewcommand{\leq}{\leqslant}
\newenvironment{refer} 
{
	\begin{list}
		{}
		{
			\setlength{\labelwidth}{.5em}
			\setlength{\leftmargin}{0.4cm}
			\setlength{\itemsep}{0cm}
		} 
	}
	{\end{list}}
%\pagenumbering{roman}
%\setcounter{page}{1}

%
\newcounter{num}
\newcommand{\exo}{\addtocounter{num}{1}\noindent{\bf{Exercice \thenum}}\\[-1mm]}
%

%
\newcommand{\donnee}[1]{\\{Donnée: } \emph{#1}}
%


%
% Math
%
\newcommand{\Real}{\mathbb R}
\newcommand{\RPlus}{\Real^{+}}
\newcommand{\norm}[1]{\left\Vert#1\right\Vert}
\newcommand{\abs}[1]{\left\vert#1\right\vert}
\newcommand{\setn}[1]{\left\{#1\right\}_{\scriptscriptstyle n \ge 1}}
\newcommand{\set}[1]{\left\{#1\right\}}
\newcommand{\seq}[1]{\left<#1\right>}
\newcommand{\eps}{\varepsilon}
\newcommand{\To}{\longrightarrow}
\newcommand{\Prob}{\rm{P}}
\newcommand{\F}{\mathcal{F}}
\newcommand{\h}{\mathcal{H}}
\newcommand{\M}{\mathcal{M}}
\newcommand{\N}{\mathcal{N}}
\newcommand{\E}{{\rm E}}
\newcommand{\Hnull}{{\rm H}_{0}}
\newcommand{\Hone}{{\rm H}_{1}}
\newcommand{\Var}{{\rm Var}}
\newcommand{\Cov}{{\rm Cov}}
\newcommand{\sign}{{\rm sign}}
\newcommand{\med}{{\rm med}}
\newcommand{\tr}{{\rm tr}}
\newcommand{\T}{{\text{\tiny \rm T}}}
\newcommand{\minf}{- \, \infty}
\newcommand{\intervalle}[4]{\mathopen{#1}#2\mathpunct{},#3\mathclose{#4}}
\newcommand{\intervalleff}[2]{\intervalle{[}{#1}{#2}{]}}
\newcommand{\intervalleof}[2]{\intervalle{]}{#1}{#2}{]}}
\newcommand{\intervallefo}[2]{\intervalle{[}{#1}{#2}{[}}
\newcommand{\intervalleoo}[2]{\intervalle{]}{#1}{#2}{[}}   
%
\newcommand{\separation}{{\begin{center}\rule{10cm}{0.25pt}\end{center}}\noindent}
%
\frenchspacing




\begin{document}
	\header{probabilités élémentaires}{Série 1}
	%
	% Exercice 1
	%
	\begin{exo}
		\donnee{Un groupe de consommateurs a réalisé une étude pour analyser le service offert par 200 employés de	divers restaurants. On s’intéresse à une possible relation entre la qualité du service et la qualification du personnel (diplômé d’une école hôtelière ou non). Les résultats de l’enquête figurent dans le tableau ci-dessous :}
		\begin{enumerate}[label=\alph*)]
			\item 
			\begin{itemize}
				\item[--] $0.455$;
				\item[--] $0.555$;
				\item[--] $0.305$.
			\end{itemize}
			\item équiprobabilité
		\end{enumerate}
	\end{exo}
	%
	% Exercice 2
	%
	\begin{exo}
		\donnee{fg}
		\begin{multicols}{3}
			\begin{enumerate}[label=\alph*), parsep=0cm, itemsep=3mm, topsep=3mm]
				\item $0.1$
				\item $0.3$
				\item $0.2$
			\end{enumerate}
		\end{multicols}
	\end{exo}
	
	%
	% Exercice 3
	%
	\begin{exo}
		\donnee{df}
		\begin{multicols}{2}
			\begin{enumerate}[label=\alph*), parsep=1mm, itemsep=3mm, topsep=3mm]
				\item\label{q.quatrieme} $\left( \dfrac{5}{6} \right)^{10}$
				\item $1 - \left( \dfrac{5}{6} \right)^{10}$ 
			\end{enumerate}
		\end{multicols}
	\end{exo}
	
	%
	% Exercice 4
	%
	\begin{exo}
		\donnee{df}
		\begin{enumerate}[label=\alph*), parsep=0cm, itemsep=3mm, topsep=3mm]
			\item $P(A) = p$ \hspace{5mm} $P(B) = 2p$ \hspace{5mm} $P(A \cup B) = 0.28$
			\item $p = 0.1$
		\end{enumerate}
	\end{exo}
	
	%
	% Exercice 5
	%
	\begin{exo}
		\donnee{df}
		\begin{enumerate}[label=\alph*), parsep=0cm, itemsep=3mm, topsep=5mm]
			\item $0.49$
			\item $0.23$
		\end{enumerate}
	\end{exo}
	
	%
	% Footer
	%
	\vfill
	\hrule
	\vspace{2mm}
	\noindent {\tiny Corrigé Etudiant - TIC} \hfill {\tt \tiny \today}
\end{document}
